\chapter*{Agradecimientos}
\markboth{Agradecimientos}{}
\phantomsection
\pagenumbering{roman}

Cuando empecé la aventura que me traje al día de hoy, allá por el año 2008, yo tenía grandes expectativas pero ningún plan concreto.
Sabía que quería estudiar en la Universidad de La Habana, y que quería hacer algo con computadoras, pero nada más.
Mi padre se sentó conmigo, él al timón, yo en el asiento del acompañante, como siempre hacíamos por esa época, esperando a que mi hermana, quien entonces estudiaba violín, terminara sus clases.

Papá me preguntó cuáles eran mis aspiraciones ahora que comenzaba la Universidad.
Él siempre recordaba la Universidad como la época de su vida donde más grande se hizo su mundo, y creo que por eso quería que la mía fuera una experiencia excepcional.
Yo le dije simplemente que quería ser el mejor. El mejor de mi clase, el mejor de toda la carrera, vaya, el mejor del mundo incluso.
Y en vez de decirme cuan ridículas eran esas expectativas, me dijo ``pues hagamos un plan para lograr eso''.

Y así lo hicimos, planeamos allí en media hora los pasos que supuestamente me iban a llevar a cumplir esas ridículas metas. Pasos que tenían que ver con dedicar no se cuántas horas a estudiar, con hacerme alumno ayudante lo antes posible, con trabajar en proyectos adicionales.
Hoy sé que no había ninguna garantía de que esos pasos me llevaran a ser el mejor del mundo, pero al menos me trajeron hasta aquí.
Y así aprendí que los sueños no están para cumplirse, están para empujarte a llegar más lejos, mucho más lejos de lo que tu pudieras imaginar, aunque muchas veces en direcciones totalmente diferentes de las que soñaste.

Este sueño lo he compartido con muchísimas personas, tantas que me es imposible agradecerles a todos por las huellas que han dejado en mi vida. Trataré de mencionar a algunos de los imprescindibles, pero pido disculpas de antemano por todos aquellos que seguro me faltarán.

En primer lugar, tengo que agradecer infinitamente a mi padre. Agradecerle por muchas cosas simples y mundanas, como enseñarme a montar bicicleta, a nadar, a disfrutar de los libros, a conducir, a comer saludable, a querer a los demás por sus defectos y no a pesar de ellos, y a soñar en grande.
Pero lo más importante que me dió mi padre no fueron ni habilidades específicas ni su filosofía de vida.
Papá siempre creyó que yo podía lograr lo que me propusiera, incluso cuando eso no coincidiera con lo que él veía más importante o valioso, incluso aunque que mis planes y expectativas me llevaran lejos de las suyas.
Y no es cierto, uno no puede lograr todo lo que se propone, claro que no.
Su propia vida estuvo llena de metas fallidas y expectativas frustradas, como la de todos.
Pero si uno no cree que puede, no tiene sentido ni siquiera intentarlo.
Y por eso el superpoder de mi padre era creer, creer que todo era posible, que solo dependía de uno mismo, y que no había obstáculo suficientemente grande.

Si bien mi padre me dio gran parte del combustible que quemé para hacer lo que he hecho, nunca lo hubiera logrado sin mi madre.
Cuando las cosas se ponían de verdad complicadas, cuando parecía que no había más solución que rendirse, mi madre siempre encontraba una salida, una forma de reinventarse para seguir adelante.
Lo hizo cuando yo era un niño y no alcanzaba la comida para los tres, lo hizo de nuevo cuando nació mi hermana, lo volvió a hacer años después cuando todo lo que tenían construido tambaleó, y después de aquel octubre de 2019 cuando la vida de todos nosotros dio un vuelco repentino y definitivo.
De mi madre aprendí muchas cosas, pero quizás la más importante es esa: que uno tiene el poder de reinventarse tantas veces como sea necesario, por uno mismo, y por los que uno quiere.
Y que nadie más que tú mismo puede realmente derrotarte.

Mi hermana ha sido la tercera constante en mi vida, o al menos la mayor parte de ella.
En muchas cosas somos muy parecidos, en otras, diametralmente opuestos.
Alenia trajo una música especial a mi mundo, una forma de ver la vida, de querer a los demás, y de simplemente ser, que complementa de manera perfecta mi propia forma de ver el mundo.
Hay una cosa en especial que compartimos, una admiración mutua desde nuestro punto de vista por el otro, que nos hace estar cerca aunque haya distancia física.
Siendo yo el mayor, siempre he sentido que mi hermana es mi fan número uno.
Lo que ella no supone, porque casi nunca lo he dicho, es que yo también soy su fan número uno, y que a pesar de los años de supuesta ventaja que le llevo, son muchas las enseñanzas que he sacado de verla crecer, convertirse en una persona capaz y feliz de cometer sus propios errores, bajo sus propias condiciones, siendo una fuente de energía vital para muchos a su alrededor, y sin perder su propia identidad en el proceso.
Y junto a Alenia, tengo que agradecer a Andy por acompañarla, por complementarla, y por traer a nuestra familia su júbilo, sus ganas de vivir, de reír, y de cantar.
Ellos dos me han enseñado el valor de perseguir tus sueños, contra viento y marea, a pesar de todas las opiniones que puedas tener en tu contra.

Mi familia no es muy grande, y no siempre han podido estar todos los que hubiesen querido. Pero de cada uno he aprendido algo, y a cada uno le dedico también un pedacito de estas páginas.
A mi abuela Sofía, que estuvo ahí desde mis primeros días de escuela, y que sigue ahí a pesar de los pesares.
A mi abuelo Roberto, que fue un ejemplo de honradez, de trabajo duro, y de entrega por sus seres queridos, y quien desgraciadamente no pudo ver a sus nietos crecer tanto como hubiese querido, pero estoy seguro estaría orgulloso.
A mi abuelo Pepín, que a pesar de lo poco que pudo compartir conmigo, guardo recuerdos hermosos. A mis tíos y primos por ambas partes de la familia, que han estado en momentos buenos, y algunos en otros momentos no tan buenos.
A mi tío Frank en especial por el cariño incondicional, y a mi primo Francito, como siempre le llamaré, por tantas horas que compartimos de niños, y las que seguro nos quedan por compartir.

Quiero agradecer también a mi segunda familia, la que no me tocaba, pero me acogió como si fuera un hijo más.
A mis suegros, Estela y Armando, que me enseñaron a ver la vida de forma un poco diferente; sus enseñanzas sí que me han marcado, mucho más de lo imaginan, y mucha buena culpa tienen de la persona que soy hoy.
A Wendy, que ha sido una hermana más, por su forma de sentir y querer que no conoce fronteras ni limitaciones; y a Daniel, por su amistad, y por darme algunas de las horas de conversación más interesantes que he tenido.
A mis abuelos adoptivos, Mimi, Gregorio, Ondina, y Armando, por su cariño immerecido; a Tony, Betty, Alain y Yahima, por hacerme parte de su familia.
Y a todos los demás, tíos, primos, y abuelos, por acogerme como uno más en su familia gigante.

Amigos tengo muchísismos, algunos desde siempre, otros desde hace poco, y todos han dejado una huella en mi.
En especial quiero agradecir a mis amigos del alma, con los que viví tres años compartiéndolo todo, desde el agua y la comida hasta las aspiraciones, los sueños, y las ganas de vivir.
A Silvio, Charly, Ian Pedro, Pepito, y Luis Alberto, no tengo forma de agradecerles por tantos años de estar ahí, en las buenas y en las malas, aunque estemos desperdigados por todos los continentes.
A mis amigos y compañeros de la UH, con los que he aprendido no solo habilidades y conocimientos, sino formas nuevas de pensar y de ver la vida.
A mis amigos de Alicante, de España, y del resto del mundo, que me han abierto los horizontes y me han dado nuevas esperanzas y expectativas.
De todos ellos me llevo un trocito, y espero que todos tengan en cambio algo de mí.
Y quiero agradecer también a los más chiquitos, que fueron mis estudiantes y ahora se han convertido en compañeros, a Jonpi, Roci, Daniel, Hian, Sadán, Estevanell, y todos los que vienen en camino.
Ellos son mi mayor fuente de orgullo, y por cada cosa que hayan podido aprender de mi, hay algo que he aprendido también yo de ellos.

Desde mis primeras letras hasta el día de hoy, todo lo que pueda haber logrado lo debo en gran medida a mis profesores, que me enseñaron a leer, a sumar, a derivar, a programar, a escribir artículos, a hablar en público, a dirigir un proyecto.
Son demasiados para nombrarlos a todos, pero sepan que los recuerdo, y si escogí esta profesión, que considero la más bella, es por la admiración que me han hecho sentir de su trabajo.
En especial, quiero agradecer a Marrero, por inspirarme su amor por la ciencia, y enseñarme la belleza que hay en entender el universo.
A Lupi, por prestarme atención en aquel primer año, y empujarme siempre a saber más.
A Bolu, por enseñarme a pensar como un científico, y animarme a seguir ese camino.
A Luciano y Katrib, por inspirarme con su energía inagotable y demostrarme que no hay cansancio que valga si uno ama su trabajo.
Por supuesto, a mis tutores de Alicante, Yoan, Rafa y Andrés, por confiar en mi, y guiarme por este viaje con infinita paciencia.
Y muy especialmente, a Yudy, por ser maestro y amigo, por enseñarme que siempre podía aspirar a más incluso aunque el resto del mundo no estuviera de acuerdo, y por estar siempre dispuesto a compartir este sueño.

Finalmente, quiero agradecer a mi esposa, mi mejor amiga, y mi inseparable compañera de equipo, Suilan.
Ella me ha enseñado demasiadas cosas para resumirlas aquí, pero la lección más importante que he aprendido es que, en última instancia, nadie más que tú puede decirte que algo es imposible.
Suilan expandió mi universo rompiendo límites que yo ni me daba cuenta que existían, y se atrevió a caminar por ese universo conmigo, compartiendo miles de horas de frustración, pero también innumerables alegrías.
Ella es mi mayor crítico, mi más valioso consejero, y el más importante miembro en todos mis equipos.
Nada de lo que pueda decir aquí sería suficiente para resaltar la importancia que ha tenido su presencia en mi vida, como una gigante roja cuya fuerza gravitacional cambia todo a años luz de distancia.
De no haberla conocido, no tengo idea de dónde estaría, pero seguro seríamos ambos muy diferentes a cómo somos.
Y por si fuera poco, me ha dado la alegría mayor de saber que pronto seremos padres, y tendremos así una nueva aventura, seguramente más difícil que todas las anteriores, pero una aventura que también haremos juntos.

Cuando empecé este viaje hace más de 13 años, nunca imaginé donde estaría hoy.
Imaginé muchas cosas, algunas que pasaron, y muchas otras que no pasaron.
Por el camino he aprendido tantas cosas, cosas sobre el mundo, y cosas sobre mí, que no estoy seguro de pueda decir que aquella persona que empezó con aquel plan en aquella tarde de 2008 sea la misma que escribe estas letras.
Muchas puertas se me abrieron, gracias a muchas personas que no he alcanzado a mencionar, así como otras puertas se cerraron.
He cruzado aquellas que consideré mejor, y aunque estoy seguro que muchas veces dejé ir oportunidades que otros veían valiosas, si tuviera la posibilidad de repetir cada una de mis decisiones trascendentales, haría exactamente lo mismo sin pensarlo dos veces.
Cualquier error que haya cometido es total responsabilidad mía, y asumo la culpa con gusto, porque son esos errores los que me han traído a poder escribir estas palabras hoy.
Cualquier ventaja, justa o injusta que haya tenido, se la debo a alguien, y por esas estoy agradecido.

Cierro con este agradecimiento general a las oportunidades que me ha dado la vida, este sueño de convertirme en un experto en el campo de estudio que escogí ejercer, y de ganarme el respeto y admiración de mis pares.
De ahora en adelante, tengo muchos sueños nuevos.
Todavía sueño con ser el mejor: el mejor padre, el mejor maestro, el mejor amigo.
Y lo sé, los sueños no están ahí para cumplirse, están para empujarte a llegar más lejos, mucho más lejos de lo que tu pudieras imaginar.

A todos los que han estado y los que siguen estando dispuestos a perseguir estos sueños conmigo, muchas gracias.

% \begin{comment}
\vspace{1cm}
\begin{flushright}
{\it Alejandro Piad Morffis}\\
{\it La Habana, a 15 de Octubre de 2021}
\end{flushright}
% \end{comment}