
\chapter[Corpus eHealth-KD 20198 \textit{A corpus to support eHealth Knowledge Discovery technologies}]{Corpus eHealth-KD 2018}
\label{Chap:CorpusV1}

Ese capítulo presenta el artículo \textit{A corpus to support eHealth Knowledge Discovery technologies}.
Este artículo introduce y describe el corpus \textit{eHealth-KD 2018}. El corpus es una colección de $1173$ oraciones relacionadas con la salud en español anotadas manualmente con una estructura semántica general que captura la mayor parte del contenido, sin recurrir a etiquetas de dominio específico. La representación semántica se define e ilustra primero con oraciones de ejemplo del corpus. Se resume además el proceso de anotación y se proporcionan métricas relevantes del corpus. Finalmente, se presentan tres implementaciones computacionales, que son compatibles con los modelos de aprendizaje automático y fueron diseñadas para estimar la complejidad del aprendizaje de la semántica del corpus. El corpus resultante se utilizó como escenario de evaluación en TASS 2018 y se discuten los resultados obtenidos por los participantes. El corpus \textit{eHealth-KD} proporciona el primer paso en el diseño de un marco semántico de propósito general que se puede utilizar para extraer conocimiento de varios dominios.

\BlankLine
\noindent \textbf{Entrada bibliográfica:}\\
Piad-Morffis, A., Gutiérrez, Y., Muñoz, R. (2019). A corpus to support ehealth knowledge discovery technologies. \textit{Journal of Biomedical Informatics}, 94, 103172.

\BlankLine
\noindent \textbf{Disponible en:} \url{https://doi.org/10.1016/j.jbi.2019.103172}
