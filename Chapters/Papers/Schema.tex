
\chapter[Esquema de Anotación: \textit{A General-Purpose Annotation Model for Knowledge Discovery: Case Study in Spanish Clinical Text}]{Esquema de Anotación}
\label{Chap:Schema}

En este Capítulo se presenta el artículo \textit{A General-Purpose Annotation Model for Knowledge Discovery: Case Study in Spanish Clinical Text}. Este artículo define un modelo de anotación diseñado para capturar una gran parte de la semántica del texto en lenguaje natural. Se presenta la estructura del modelo de anotación, con ejemplos de oraciones anotadas y una breve descripción de cada rol semántico y relación definida. Esta investigación se centra en una aplicación a textos clínicos en español. Sin embargo, el modelo de anotación presentado es extensible a otros dominios e idiomas. También se proporciona un ejemplo de oraciones anotadas, una guía de anotación y archivos de configuración adecuados para una herramienta de anotación usara por la comunidad científica.

\BlankLine
\noindent \textbf{Entrada bibliográfica:}\\
Piad-Morffis, A., Guitérrez, Y., Estevez-Velarde, S., Muñoz, R. (2019, June). A general-purpose annotation model for knowledge discovery: Case study in Spanish clinical text. In \textit{Proceedings of the 2nd Clinical Natural Language Processing Workshop} (pp. 79-88).

\BlankLine
\noindent \textbf{Disponible en:} \url{http://dx.doi.org/10.18653/v1/W19-1910}
