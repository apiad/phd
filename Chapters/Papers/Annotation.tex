
\chapter[Anotación Asistida: \textit{Active Learning for Assisted Corpus Construction:A Case Study in Knowledge Discovery from Biomedical Text}]{Anotación Asistida}
\label{Chap:Annotation}

Este capítulo presenta el artículo \textit{Active Learning for Assisted Corpus Construction:A Case Study in Knowledge Discovery from Biomedical Text}.
Este artículo define un enfoque de aprendizaje activo que tiene como objetivo reducir el esfuerzo humano requerido durante la anotación de corpus en lenguaje natural, compuestos por entidades y relaciones semánticas. Nuestro enfoque ayuda a los anotadores humanos seleccionando inteligentemente las oraciones más informativas para anotar y pre-anotándolas con algunas entidades y relaciones semánticas. Se define una estrategia basada en la incertidumbre con un factor de densidad ponderado, utilizando métricas de similitud basadas en \textit{embeddings} de oraciones. Los resultados experimentales sugieren que la estrategia de consulta reduce entre un $35\%$ y $40\%$ el número de oraciones que se deben anotar manualmente para desarrollar sistemas capaces de alcanzar un resultado objetivo de $F_1$,
mientras que la estrategia de pre-anotación produce una reducción adicional de $24\%$ en el tiempo total de anotación.

\BlankLine
\noindent \textbf{Entrada bibliográfica:}\\
Hian Cañizares-Díaz, Alejandro Piad-Morffis, Suilan Estevez-Velarde, Yoan Gutiérrez, Yudivián
Almeida Cruz, Andrés Montoyo and Rafael Muñoz-Guillena. Active Learning for Assisted Corpus Construction: A Case Study in Knowledge Discovery from Biomedical Text. \textit{Proceedings of RANLP 2021.}

\BlankLine
\noindent \textbf{Disponible en:} \url{https://ranlp.org/ranlp2021/proceedings-20Sep.pdf}
