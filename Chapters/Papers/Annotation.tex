
\chapter[Anotación Asistida: \textit{Active Learning for Assisted Corpus Construction:A Case Study in Knowledge Discovery from Biomedical Text}]{Anotación Asistida}
\label{Chap:Annotation}

Este capítulo presenta el artículo \textit{Active Learning for Assisted Corpus Construction:A Case Study in Knowledge Discovery from Biomedical Text}.
Este artículo define un enfoque de aprendizaje activo que tiene como objetivo reducir el esfuerzo humano requerido durante la anotación de corpus en lenguaje natural, compuestos por entidades y relaciones semánticas. Nuestro enfoque ayuda a los anotadores humanos seleccionando inteligentemente las oraciones más informativas para anotar y pre-anotándolas con algunas entidades y relaciones semánticas. Se define una estrategia basada en la incertidumbre con un factor de densidad ponderado, utilizando métricas de similitud basadas en \textit{embeddings} de oraciones. Como caso de estudio, se evalúa este enfoque mediante simulación en el corpus \textit{eHealth-KD 2019} estimando la reducción potencial en el tiempo de anotación total. Los resultados experimentales sugieren que la estrategia de consulta reduce entre un $35\%$ y $40\%$ el número de oraciones que se deben anotar manualmente para desarrollar sistemas capaces de alcanzar un resultado objetivo de $F_1$,
mientras que la estrategia de pre-anotación produce una reducción adicional de $24\%$ en el tiempo total de anotación. En general, los experimentos preliminares sugieren que se podrían ahorrar hasta $60\%$ del tiempo de anotación produciendo corpus que tienen la misma utilidad para entrenar algoritmos de aprendizaje automático.
Aunque el artículo es de autoría compartida, representa un resultado directo de la investigación presentada en esta Tesis, y por tal motivo se incluye en el compendio.
El artículo fue aceptado para publicación en las memorias del evento \textit{Recent Advances in Natural Language Processing, 2021}.