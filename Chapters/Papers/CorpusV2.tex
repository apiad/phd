
\chapter[Ecosistema Computacional: \textit{A Computational Ecosystem to Support eHealth Knowledge Discovery Technologies in Spanish}]{Ecosistema Computacional}
\label{Chap:CorpusV2}

Este capítulo presenta el artículo \textit{A Computational Ecosystem to Support eHealth Knowledge Discovery Technologies in Spanish}. Este trabajo describe un ecosistema que facilita la investigación y el desarrollo en el descubrimiento del conocimiento en el dominio biomédico, específicamente en el idioma español. Con este fin, se desarrollan y comparten varios recursos con la comunidad de investigación, incluido un nuevo modelo de anotación semántica, un corpus anotado de 1.045 oraciones y recursos computacionales para construir y evaluar técnicas automáticas de descubrimiento de conocimiento. Además, se define una tarea de aprendizaje automático con criterios de evaluación objetivos, y se diseña un entorno de evaluación en línea, lo que permite a los investigadores interesados en esta tarea obtener resultados inmediatos y compararlos con el estado del arte. Como caso de estudio, se analizan los resultados de un evento competitivo basado en estos recursos y se brindan pautas para futuras investigaciones.

\BlankLine
\noindent \textbf{Entrada bibliográfica:}\\
Piad-Morffis, A., Gutiérrez, Y., Almeida-Cruz, Y., Muñoz, R. (2020). A computational ecosystem to support eHealth Knowledge Discovery technologies in Spanish. \textit{Journal of Biomedical Informatics}, 109, 103517.

\BlankLine
\noindent \textbf{Disponible en:} \url{https://doi.org/10.1016/j.jbi.2020.103517}
