\chapter*{Abstract}

The increasing amount of information published online presents a significant challenge for the scientific community. The availability of these resources makes it possible to accelerate research in multiple branches of science, by connecting results from different research groups. However, the volume of information produced is impossible to process by humans in its entirety, hence, the scientific community wastes time and resources in rediscovering the same results, due to lack of communication. The application of artificial intelligence techniques allows the construction of computer systems that help researchers to search, analyse and connect the existing information in large volumes of data. This process is called automatic knowledge discovery and it is a branch of research with increasing interest.

The ehealth domain is one of the scenarios in which automatic knowledge discovery can produce a greater impact for the benefit of society. The recent COVID-19 pandemic is an example where the production of scientific articles has far exceeded the capacity of the scientific community to assimilate them. To mitigate this phenomenon, linguistic resources have been published, useful for building automatic knowledge discovery systems.
However, knowledge discovery requires not only linguistic resources, it needs computational resources and available infrastructure to systematically evaluate results and objectively compare alternative approaches.

This work describes an ecosystem that facilitates research and development in automatic knowledge discovery in the biomedical domain, specifically in the Spanish language, although it can be extended to other domains and languages. To this end, several resources are developed and shared with the research community, including a new semantic annotation model, three corpora with more than $2,700$ sentences, and $34,000$ manually performed semantic annotations, as well as computational resources to build and evaluate techniques for automatic knowledge discovery.
These resources include baseline implementations of knowledge discovery algorithms that serve as the basis for building more advanced solutions.
In addition, a research task is defined with objective evaluation criteria and an online evaluation environment is configured and maintained that allows researchers interested in this task to obtain immediate feedback and compare their results with the state of the art.
As a case study, the results of several teams of researchers in three consecutive editions of a competitive challenge organised based on these resources are analysed.

Based on the experiences obtained during the manual annotation process, an assisted annotation strategy is designed that produces a considerable reduction in human annotation time.
The approach helps human annotators by intelligently selecting the most informative sentences to annotate and then pre-annotating them with some highly accurate semantic entities and relationships.
This strategy is evaluated in the corpus developed in this research, and is published in the form of a computational tool available to the scientific community.

This computational ecosystem provides an effective learning and assessment environment to foster research in knowledge discovery in both biomedical documents and other domains. The annotated corpus can be used to train and evaluate computational knowledge discovery systems, and compare them automatically with the state of the art. Likewise, the computational tools developed can be used to build new systems and to create new linguistic resources in other languages or domains. All the resources developed in this research are publicly available for use by the scientific community\footnote{\url{https://ehealthkd.github.io}}.