\chapter*{Preface}
\markboth{Preface}{}
% \addcontentsline{toc}{chapter}{Preface} % - It has to be at the beginning; otherwise, indexes the last page
% \pagenumbering{roman}

{\rightline{{\it `[...] who was six years old on 6/6/66 [...]'}}}
\vspace{2\baselineskip}
\noindent \lettrine[lines=2, findent=1pt, nindent=0pt]{W}{ith} such particular sentence Frank Zappa's guitar solos transcription book titled {\it The Frank Zappa Guitar Book} began~\cite{Book:ZappaVai:1982}. Nevertheless, this quotation was not about Frank as it related to the other author of the book, who found really curious this unusual fact historically related to the Devil or the Antichrist. This person was an unknown musician at the time, but he was (and still is) one of most innovative electric guitar players ever. His name was Steven Siro Vai and, despite his concerns about the {\it evilness} of the birth date, he later claimed they all disappeared once he met Marilyn Manson.

%{\it `[...] who was six years old on 6/6/66 [...]'}~\cite{Book:ZappaVai:1982}. With such particular sentence Frank Zappa's guitar solos transcription book titled {\it The Frank Zappa Guitar Book} began. Nevertheless, this quotation was not about Frank as it related to the other author of the book, who found really curious this unusual fact historically related to the Devil or the Antichrist. This person was an unknown musician at the time, but he was (and still is) one of most innovative electric guitar players ever. His name was Steven Siro Vai and, despite his concerns about the {\it evilness} of the birth date, he later claimed they all disappeared once he met Marilyn Manson.

%\lettrine[lines=3, findent=3pt, nindent=0pt]{{\it `W}}{{\it ho}} was six years old on 6/6/66'~\cite{Book:ZappaVai:1982}. With such particular sentence Frank Zappa's guitar solos transcription book titled {\it The Frank Zappa Guitar Book} began. Nevertheless, this quotation was not about Frank as it related to the other author of the book, who found really curious this unusual fact historically related to the Devil or the Antichrist. This person was an unknown musician at the time, but he was (and still is) one of most innovative electric guitar players ever. His name was Steven Siro Vai and, despite his concerns about the {\it evilness} of the birth date, he later claimed they all disappeared once he met Marilyn Manson.

Steve Vai grew up in Long Island (New York) and began playing guitar at the age of 13. His first guitar teacher, also an unknown musician at the time, was only three years older than him but with a great reputation in the area. His name was Joe Satriani and he is nowadays considered one of the main references in electric guitar playing. It is told that, the day of the first class, Steve went to Joe's house (where he would receive the lessons) carrying an acoustic guitar without any strings on it. Nevertheless, as lessons kept on going, Steve became impressed with the possibilities of that instrument. Fallen in love with it, Steve began to practise as much as he could, which eventually led to a great development of both a great technical ability and remarkable musical skills.

As a teenager Steve became literally obsessed with the music of Frank Zappa, up to the point of being totally determined to become a member of his band. At the age of twenty, while attending to the prestigious Berklee School of Music in Boston (Massachussets), Steve sent Frank some material he thought the well-known musician would be interested in. Apart from several tapes showing off his very skilled and mature sense of guitar playing, there was a transcription of one of Zappa's most challenging pieces called {\it The Black Page}. This piece, composed for the drum kit and originally performed by the renowned drummer Terry Bozzio, receives its name due to the large amount of notes, ornamentations and annotations present in it, which makes the score resemble a black page. Frank, impressed with the talent and abilities of that unknown musician, immediately hired him.

Steve left Berklee and started his career in Frank Zappa's band. Initially, most of his duties consisted in transcribing music, typically guitar solos and drum sections. After some time he became a full-time band member, often playing {\it impossible} guitar parts credited as {\it Strat Abuse}. Eventually, some years later he left the band to pursue his own musical career, becoming one of the most acclaimed electric guitar players nowadays.

And, how does this story relate to this work? In general, distinguishing the different instruments, notes, rhythms, chords, tonalities is not a trivial task which certainly requires a great deal of practise. On top of that, people with such skills who also have the necessary expertise to represent this information in an abstract musical format is definitely scarce. Nevertheless, symbolic music representations and codifications of the information present in audio streams are undoubtedly useful for tasks such as preservation, reproducibility, or musicological analysis, among many others.

This dissertation focuses on this issue of retrieving a symbolic high-level representation which abstracts the musical information present in an audio piece using computational approaches. This process is known as Automatic Music Transcription in the Music Information Retrieval community in which it shows large application, not only as a user-end application but also as an intermediate process for other tasks.

Nevertheless, due to its high complexity, this problem is still far from being solved, reason why Frank would still probably require Steve for transcribing some of his improvisations. However, small research contributions as this work should eventually provide more accurate and reliable techniques not only applicable in the research community but also on a daily basis.
